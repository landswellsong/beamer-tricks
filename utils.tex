% Assuming documentclass=beamer

% XeLaTeX stuff
\usepackage{fontspec}
\usepackage{xltxtra}
\usepackage{xunicode}
\setsansfont{Ubuntu}
\newfontfamily{\cyrillicfonttt}{Ubuntu Mono}
\newfontfamily{\cyrillicfont}{Ubuntu}
% TODO: mono font
% TODO: stress marks

% Code listings
\usepackage{minted}
\newminted{py}{escapeinside=||,linenos=true} % magic!

% Arrows et al
\usepackage{tikz}
\usetikzlibrary{tikzmark, shadows, shapes.callouts, calc}
% http://tex.stackexchange.com/questions/58188/beamer-source-code-highlighting-annotation-tips
\tikzset{notice/.style  = { draw, rectangle callout, callout absolute pointer={#1}, rounded corners,fill=gray!20,drop shadow,font=\footnotesize} }

% Environment for making tikz stuff invisible
\newenvironment{tikzstuff}
    {\setbeamercovered{invisible}\begin{tikzpicture}[remember picture, overlay]
    \coordinate (belowright) at (0.5, -0.7); % TODO: above versions should point slightly above the actual point
    \coordinate (aboveright) at (0.5, 0.7);  % hint: make 2 styles "above" and "below" or 2 commands 
    \coordinate (belowleft) at (-0.5, -0.7);
    \coordinate (aboveleft) at (-0.5, 0.7);}
    {\end{tikzpicture}\setbeamercovered{transparent}}

% Place a tikz callout at anchor #1, direction #2 (see above), and text #3
\newcommand{\callout}[3]{\node[notice={(pic cs:#1)}] at ($(pic cs:#1) + (#2)$) {#3};}

% Shortcut for tikzmark
\newcommand{\tm}[1]{\tikzmark{#1}}

% Make footnotes smaller
\setbeamerfont{footnote}{size=\tiny}

% Color the URL links but don't interfere with Beamer otherwise
\hypersetup{colorlinks,linkcolor=}

% Word wrap
\usepackage{polyglossia}
\setdefaultlanguage{ukrainian}
\setotherlanguages{english,russian}
